% Options for packages loaded elsewhere
\PassOptionsToPackage{unicode}{hyperref}
\PassOptionsToPackage{hyphens}{url}
%
\documentclass[
  11pt,
]{article}
\usepackage{amsmath,amssymb}
\usepackage{iftex}
\ifPDFTeX
  \usepackage[T1]{fontenc}
  \usepackage[utf8]{inputenc}
  \usepackage{textcomp} % provide euro and other symbols
\else % if luatex or xetex
  \usepackage{unicode-math} % this also loads fontspec
  \defaultfontfeatures{Scale=MatchLowercase}
  \defaultfontfeatures[\rmfamily]{Ligatures=TeX,Scale=1}
\fi
\usepackage{lmodern}
\ifPDFTeX\else
  % xetex/luatex font selection
\fi
% Use upquote if available, for straight quotes in verbatim environments
\IfFileExists{upquote.sty}{\usepackage{upquote}}{}
\IfFileExists{microtype.sty}{% use microtype if available
  \usepackage[]{microtype}
  \UseMicrotypeSet[protrusion]{basicmath} % disable protrusion for tt fonts
}{}
\makeatletter
\@ifundefined{KOMAClassName}{% if non-KOMA class
  \IfFileExists{parskip.sty}{%
    \usepackage{parskip}
  }{% else
    \setlength{\parindent}{0pt}
    \setlength{\parskip}{6pt plus 2pt minus 1pt}}
}{% if KOMA class
  \KOMAoptions{parskip=half}}
\makeatother
\usepackage{xcolor}
\usepackage[margin=2.5cm]{geometry}
\usepackage{graphicx}
\makeatletter
\def\maxwidth{\ifdim\Gin@nat@width>\linewidth\linewidth\else\Gin@nat@width\fi}
\def\maxheight{\ifdim\Gin@nat@height>\textheight\textheight\else\Gin@nat@height\fi}
\makeatother
% Scale images if necessary, so that they will not overflow the page
% margins by default, and it is still possible to overwrite the defaults
% using explicit options in \includegraphics[width, height, ...]{}
\setkeys{Gin}{width=\maxwidth,height=\maxheight,keepaspectratio}
% Set default figure placement to htbp
\makeatletter
\def\fps@figure{htbp}
\makeatother
\setlength{\emergencystretch}{3em} % prevent overfull lines
\providecommand{\tightlist}{%
  \setlength{\itemsep}{0pt}\setlength{\parskip}{0pt}}
\setcounter{secnumdepth}{5}
\usepackage{multicol}
\usepackage{abstract}
\renewcommand{\abstractname}{\large\textbf{Abstract}}
\ifLuaTeX
  \usepackage{selnolig}  % disable illegal ligatures
\fi
\usepackage{bookmark}
\IfFileExists{xurl.sty}{\usepackage{xurl}}{} % add URL line breaks if available
\urlstyle{same}
\hypersetup{
  pdftitle={Modellazione statistica e grafica su GSS},
  pdfauthor={Riccardo Fantechi},
  hidelinks,
  pdfcreator={LaTeX via pandoc}}

\title{Modellazione statistica e grafica su GSS}
\author{Riccardo Fantechi}
\date{\textit{Università degli Studi di Firenze}}

\begin{document}
\maketitle

{
\setcounter{tocdepth}{2}
\tableofcontents
}
\begin{center}
\hrulefill
\end{center}

\vspace{0.8em}

\begin{abstract}
In questa analisi statistica esploriamo le opinioni degli intervistati del dataset GSS riguardo la regolamentazione delle armi da fuoco. L’obiettivo principale è costruire e confrontare modelli grafici e predittivi, al fine di comprendere quali fattori sociodemografici influenzano il supporto a leggi restrittive come GUNLAW.
\end{abstract}

\begin{multicols}{2}

# Introduzione

In questa analisi andremo a studiare il comportamento della variabile 

# Pulizia e Preprocessing del dataset

## Caricamento del dataset

## Rimozione dei valori mancanti

## Codifica delle variabili in fattori

# Analisi esplorativa

## Statistiche descrittive

## Tabelle di contingenza e percentuali

## Visualizzazione grafica con Heatmap e Barplot

# Modelli grafici indiretti

## Introduzione ai modelli log-lineari

## Costruzione del modello 

## Interpretazione delle dipendenze

# Modelli grafici diretti (Reti Bayesiane)

## Introduzione alle Reti Bayesiane

## Algoritmo Hill climbing

## Interpretazione

# Regressione Logistica per Confronto

## Specificazione del modello

## Analisi dei coefficienti

## Interpretazione dei risultati

# Selezione del Modello (AIC e BIC)

## Confronto tra modelli grafici

## Confronto tra reti bayesiane

## Confronto con regressione logistica

# Conclusioni e Interpretazioni Finali

## Sintesi dei risultati

## Limiti dell’analisi

## Spunti per analisi future

\end{multicols}

\end{document}
